\chapter{Introduction}
% \pagenumbering{arabic}

\section{Motivation}

Excessive speeding poses a significant threat to the safety of people on our roads. It impacts those directly involved in accidents, their families, and their communities. Over four years, from 2019 to 2022, Germany experienced an average of nearly 38,000 accidents annually due to speeding \cite{Statis_2023b}. These accidents can have long-lasting consequences, inflicting physical and emotional harm. Countless individuals are injured or tragically losing their lives, profoundly impacting their loved ones. It is crucial to address this issue, and the goal of this thesis is to develop an innovative speed gun system to aid in the reduction of excessive speeding.

\section{Problem Statement}
Law enforcement agencies and road safety organizations often find the current speed gun technologies available in the market too expensive for widespread adoption. These technologies can cost around 2000 €, which creates a financial barrier preventing effective speed monitoring in many regions. This thesis aims to explore alternative speed gun systems that use computer vision to address this challenge. By using this technology, we aim to create a more affordable and robust solution for speed monitoring, which will address the financial constraints of traditional speed guns and open up new possibilities for innovation and customization in speed monitoring technology. Ultimately, this will enhance road safety and reduce accidents caused by excessive speed.

\section{Previous Work}
Previous research \cite{Sabtu_2023} has delved into various computer vision methods to tackle the complex challenge of speed monitoring. One remarkable innovation is the image alignment algorithm, which employs feature detection techniques to fix unintended movement during video recording. This algorithm enhances the stability of recorded footage and ensures accurate speed assessment.

Another technique explored is optical flow analysis, which is a method that tracks objects in a video sequence by calculating the velocity of points within the images and estimating where those points could be in the following sequence. This technique is computationally efficient, making it a practical and cost-effective solution for real-time speed monitoring.

Lastly, a method to measure object distance using the pinhole camera model was also explored. This model leverages the pinhole camera's geometry to calculate the distance of objects from the camera's vantage point. Integrating this model into speed monitoring provides a streamlined and economically viable method for assessing speeds with high precision.

After conducting thorough testing, it has been established that these techniques exhibit a high level of effectiveness in monitoring speed, achieving an impressive accuracy rate of 94 \%. This outcome signifies their viability for advancing to the next stage of development and potential incorporation into a prototype speed gun system.

\section{Research Objectives}
Our research aims to develop an alternative to current implementation of speed gun by utilizing computer vision technology. The work will consist of two parts. In the first part, we will design the physical structure of the prototype using the VDI 2221 methodology. The second part will focus on software development, creating an easy-to-use interface that leverages computer vision for real-time speed assessment and data analysis. The goal is to create a state-of-the-art speed gun that addresses speeding concerns and sets a new road safety technology standard.

\chapter{State of the Art - Speed Pistol}
\label{chap:stateoftheart}
Law enforcement agencies worldwide utilize speed guns, also known as radar guns or speed pistols, as crucial tools to combat speeding. These devices play a pivotal role in maintaining road safety by accurately measuring the velocity of vehicles on the road \cite{Hull_2020}.

The current implementation of speed pistol usually utilize either the LIDAR (Light Detection and Ranging) or RADAR (Radio Detection and Ranging) technology \cite{StalkerRadar_2023} \cite{FlyGuys_2023} \cite{Kustom23} \cite{LaserTech_2023}. These technologies serve as the cornerstone of modern speed measurement devices, allowing law enforcement officers to gauge the speed of vehicles in real-time accurately.

LIDAR technology operates on emitting laser pulses towards a target vehicle and measuring the time it takes for the light to bounce back \cite{FlyGuys_2023}. The LIDAR device can precisely calculate the vehicle's speed by analyzing the returned signals.

On the other hand, RADAR technology has been a reliable tool in speed enforcement for decades. Radar guns emit radio waves as a focused beam, which bounce off the target vehicle and return to the device. By analyzing the frequency shift of the returned signal, the device can determine the vehicle's speed \cite{Policeradar}.

However, both of these technologies have their limitations. For instance, LIDAR can be affected by adverse weather conditions like rain or fog, potentially reducing its effectiveness \cite{dressig}. Conversely, RADAR signals may be susceptible to interference from nearby objects, which can complicate speed measurements in densely populated areas \cite{Hossain}.

Both LIDAR and RADAR technologies, while highly effective in speed measurement, come with a notable price tag. A LIDAR device can range from several thousand to tens of thousands of euros \cite{ProLaser4}. In comparison, RADAR guns, though generally less expensive than LIDAR, can still cost around 2000 € for a high-quality unit \cite{Danasafety}.

These costs present a significant consideration for law enforcement agencies, especially those operating with limited budgets or in smaller jurisdictions. This limitation highlights the need for exploring more cost-effective alternatives without compromising accuracy and efficiency in speed enforcement.

