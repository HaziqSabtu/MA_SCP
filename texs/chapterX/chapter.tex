\chapter{Methodology}

\section{MVC Pattern}
The Model-View-Controller (MVC) pattern is a software architectural pattern that separates an application into three interconnected components: the model, the view, and the controller. The model represents the data and logic of the application, the view displays the data to the user, and the controller handles user input and updates the model and view accordingly. This pattern promotes separation of concerns, modularity, and code reusability in software development. \cite{Verstehen19}

\section{Design Patterns - Thread Pool}
A thread pool is a software design pattern that manages a pool of worker threads to efficiently execute tasks. Instead of creating a new thread for each task, a thread pool reuses existing threads, minimizing the overhead of thread creation. It improves performance and resource utilization by limiting the number of concurrent threads and providing a queue to handle incoming tasks.\cite{Brownlee22}

\chapter{Designing}
\section{Requirement}

Must have:
\begin{itemize}
	\item Usability - Easy to use
	\item performance - Fast processing by utilising multiple threads
	\item Responsiveness - Responsive GUI, avoid methods that block the GUI thread
	\item Error Handling - Handle errors gracefully, avoid crashing the application
	\item Scalability - For future development
	\item Documentation - user guides, Tooltips, comments
	\item Design - Clean and simple design
\end{itemize}

\section{Wireframe}
Program flow and GUI design will be presented in a wireframe.

* Flow of the program is not finalized, will be updated in the future

\section{GUI Design}
Design of the GUI will be presented here. Panels, Buttons, Textfields, etc.

\chapter{GUI-Implementation}
\section{Model}
Implementation of the model
\section{View}
Implementation of the view

\chapter{Testing}
\section{Unit Testing}
Unit testing is a software testing approach that involves testing individual components or units of code in isolation to ensure they function correctly. It verifies the behavior of small, independent units of code, such as functions or methods, to validate their expected functionality and catch any defects early in the development process. \cite{Hamilton23}

\chapter{Conclusion}
Conclusion of the project