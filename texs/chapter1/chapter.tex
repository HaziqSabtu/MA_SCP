\chapter{Introduction}
\pagenumbering{arabic}
Speeding is a significant problem on roads and highways worldwide, causing accidents, injuries, and fatalities. According to the National Highway Traffic Safety Administration (NHTSA), the trend of speeding-related accidents in the United States of America has fluctuated over the years. For example, the NHTSA reported a maximum of approximately 13,000 speeding-related fatalities in 2006. Since then, the number has gradually decreased, with an average of around 10,500 speeding-related fatalities between 2006 and 2020 \cite{carlier22}. Nonetheless, the continued prevalence of speeding-related accidents highlights the need for practical solutions to address this problem.

Traditional speed cameras are widely used to detect and deter speeding, but they have limitations, such as high costs, limited coverage, and the need for manual intervention. In recent years, computer vision techniques have shown promising results in speed detection, offering advantages such as low costs, high accuracy, and real-time processing. This paper proposes a speed camera algorithm using Raspberry Pi and computer vision, which aims to overcome the limitations of traditional speed cameras and improve speed detection accuracy and efficiency.

The proposed algorithm consists of three main components: image alignment, object detection, and speed calculation. The image alignment algorithm is necessary to align the images taken by the camera to the same reference frame. The object detection algorithm detects the vehicle in the image and tracks its movement to calculate its speed. The speed calculation algorithm calculates the vehicle's speed based on the distance traveled and the time taken. The proposed algorithm combines these three components to detect and calculate the vehicle's speed.

The methodology of the proposed algorithm is explained in detail, including the hardware and software used in the implementation. The results obtained from the implementation show high accuracy and effectiveness in detecting and calculating vehicle speed. The proposed algorithm is compared with values acquired experimentally, and the discussion highlights the strengths and weaknesses of the proposed algorithm and suggests further improvements and research.

This paper contributes to speed measurement by proposing a low-cost, high-accuracy, and real-time speed camera algorithm using Raspberry Pi and computer vision. The practical applications and significance of the proposed algorithm are also discussed, highlighting its potential to improve road safety and reduce the incidence of speeding.