\chapter{Methodology}

\section{Design Methodology}
Explaination of the design methodology from VDI 2221 \cite{Klaus13}

\begin{itemize}
    \item What is VDI 2221 and what are its key principles?
    \item What are the main objectives and goals of VDI 2221?
    \item What are the key stages or phases outlined in VDI 2221?
\end{itemize}


\chapter{Task Clarification and Specification}
\section{Requirement of the Prototype}
\label{sec:requirement}
List of requirements for the prototype

Must have:
\begin{itemize}
    \item Ergonomic - Comfortable to hold, Easy to use, Weight distributed evenly
    \item Portable - Lightweight, Small
    \item Size (MAX)
          \begin{itemize}
              \item Length: 25 cm
              \item Width: 25 cm
              \item Height: 25 cm
          \end{itemize}
    \item Weight (MAX): 3 kg
    \item Compliance and Regulation - Comply with the regulations of the country of use
    \item Cost - Affordable, < 300 Euro (including Pi, Camera and Screen)
    \item Appointments - Completed within 3 months
    \item Design - Components are packed in a chasis
    \item Camera - Camera must be presented in the prototype
    \item Power - Battery powered, Rechargeable battery, Duration min. 1 hour
    \item Control - Control via touch screen
\end{itemize}

Optional Requirements:
\begin{itemize}
    \item Durability - Water resistance, Dust resistance
    \item Modular - Easy to assemble and disassemble, Swappable parts
    \item Features - Mountable on a tripod
    \item Fertigung - 3D printed parts
\end{itemize}

\section{Requirement List}
List of requirements will be generated from the must have and optional requirements (Section \ref{sec:requirement})
\chapter{Concept Generation}
\section{Abstraction}
\begin{itemize}
    \item What is Abstraction?
    \item How does it defined and utilized in the design process?
    \item What are the benefits of using abstraction?
\end{itemize}
\section{Function Structure}
\begin{itemize}
    \item What is a function structure?
    \item What is Black Box Method?
    \item Define the function structure of the prototype using the Black Box Method according to the requirements specified.
\end{itemize}

\section{Idea Generation}
This section will discuss the methods used for idea generation.


Methods used:
\begin{itemize}
    \item Market Research
    \item Competitive Analysis
    \item Brainstorming
\end{itemize}

Method is suitable, due to the face that handheld devices are common in the market
\section{Combination of Ideas with Morphological Chart}
List of ideas from brainstorming will be combined with the function structure to generate a morphological chart

Atleast 3 Design Concepts will be generated from the morphological chart

\chapter{Design}
\section{Concept Selection and Evaluation}
\begin{itemize}
    \item Explaination of the design and discussion of advantages and disadvantages
    \item What are the performance characteristics and limitations of each design option, and how do they align with the desired outcomes?
    \item What are the cost implications associated with each design option?
    \item What are the potential risks and uncertainties associated with each design option, and how can they be mitigated or managed effectively?
\end{itemize}
\subsection{Design 1}
\subsection{Design 2}
\subsection{Design 3}

\section{Final Design}
\begin{itemize}
    \item How is the final design selected?
    \item What methods are used to evaluate the final design?
    \item Which evaluation criteria are being used?
    \item How well does each design option fulfill the functional requirements specified in VDI 2221?
\end{itemize}
\subsection{CAD Drawing}
Final CAD Design will be presented here.
Including with the features
\subsection{Bill of Materials}
List of parts used in the final design

\chapter{Conclusion}
Conclusion of the project