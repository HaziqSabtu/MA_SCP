\chapter{Conceptual Design}
Following the clarification of the task is the conceptual design, where in this section of the product development process, designers engage in creative exploration and evaluation of various design ideas and concepts.

Pahl and Beitz describe conceptual design as the phase of the design process where the essential problems are identified through abstraction, function structures are established, appropriate working principles are sought, and these elements are combined to form a working structure. This process lays down the foundation for the solution path by elaborating on a solution principle, ultimately specifying the principle solution. \cite{Pahl07d}

Figure \ref{} shows the steps involved in the conceptual design phase.


\section{Abstraction}
Traditional solution principles or designs may not provide optimal answers in the presence of new technologies, procedures, materials, and scientific discoveries. Preconceptions, conventions, and risk aversion often hinder unconventional but better and more cost-effective solutions. To overcome fixation on conventional ideas, designers utilize abstraction, focusing on the general and essential aspects rather than particular details. By formulating the task appropriately, the overall function and essential constraints become clear, enabling objective solution selection. \cite{Pahl07b}

To help in identification of the essential problems, following abstraction techniques are used:
\begin{itemize}
    \item Step 1 - Eliminate personal preferences.
    \item Step 2 - Omit requirements that have no direct bearing on the function and the essential constraints.
    \item Step 3 - Transform quantitative into qualitative data and reduce them to essential
          statements.
    \item Step 4 - As far as it is purposeful, generalise the results of the previous step.
    \item Step 5 - Formulate the problem in solution-neutral terms. \cite{Pahl07c}
\end{itemize}

Figure \ref{} shows the result of the abstraction process.

\section{Function Structures}
In the design process, a function refers to a specific action or purpose that a product or system should fulfill. It captures the essence of what the product or system is intended to do, focusing on the desired outcomes rather than specific solutions or components. Functions serve as the building blocks of a design, providing a clear understanding of the overall purpose and functionality of the intended product or system.

The function structure is a graphical representation of the functions of a system and their interrelationships. It is a hierarchical representation of the functions of a system, starting with the overall function and breaking it down into sub-functions. The function structure is a useful tool for identifying the essential functions of a system and for identifying the relationships between the functions. \cite{Pahl07e}

Figure \ref{} shows the representation of the function structure and the process of breaking down the overall function into sub-functions.


\subsection{Overall Function}
Based on the result of abstraction, the overall function of the system can be represented and visualized using a function structure diagram. This diagram, as shown in Figure \ref{}, shows the overall function, which will be broken down into sub-functions in the next step.

\subsection{Sub-Functions}
Subsequently, after the abstraction process, the overall function is further decomposed into several sub-functions. This division is guided by the function structure, which represents the interconnections and dependencies between the functions. By analyzing the main flow of the system and considering the desired outcomes, the sub-functions are identified and organized within the function structure.

The purpose of this decomposition is to reduce the complexity of the overall system and facilitate the identification of suitable solution principles that can fulfill the required functions. By breaking down the main function into smaller, more manageable sub-functions, designers can focus on addressing specific aspects of the system's functionality and finding appropriate design solutions for each sub-function.

It is important to note that a simple and straightforward structure is desirable in the function structure. Such simplicity often leads to the development of uncomplicated and cost-effective systems. By keeping the structure of the function hierarchy straightforward and easy to understand, the design process can be streamlined, and potential complexities can be minimized.

\section{Developing Working Principles}

In the process of developing working structures, one crucial step is to search for working principles. Working principles refer to the physical effects and characteristics that fulfill specific functions of the structure being designed. These principles are combined to create the working structure, and they encompass both the physical processes and the form design features.

The search for working principles aims to generate several potential solution variants, creating what is known as a solution field. This can be achieved by varying the physical effects and form design features. Often, multiple physical effects are involved in fulfilling a single subfunction or even multiple function carriers. \cite{Pahl07f}

In developing working principles, there are multiple available methods in idea generation. These methods are categorized into three groups:

\begin{itemize}
    \item Conventional methods
    \item Intuitive methods
    \item Discursive methods
\end{itemize}

Conventional methods involve a systematic and data-driven approach. Designers gather information from various sources, such as literature, trade publications, and competitor catalogs, to stay informed about advancements and best practices. They analyze natural systems and existing technical systems to draw inspiration and identify opportunities for improvement. Analogies are used to substitute analogous problems or systems, leading to fresh perspectives. Additionally, empirical studies, such as measurements and model tests, provide tangible data for validating designs and predicting real-world performance. \cite{Pahl07g}

On the other hand, intuitive methods tap into creativity and associative thinking. Brainstorming fosters a collaborative environment where diverse perspectives generate a wide range of ideas without judgment. Method 635 adds structure to brainstorming, allowing for systematic idea development within a group. The Gallery Method combines individual work with group discussions, using sketches or drawings to explore solution proposals visually. Synectics involves combining apparently unrelated concepts to trigger new and fruitful ideas. \cite{Pahl07h}

Discursive methods provide systematic step-by-step procedures influenced by intuition and creativity. They involve deliberate analysis of physical processes, leading to multiple solution variants derived from the relationships between variables. This approach fosters a deeper understanding of the problem space, encouraging the discovery of novel solutions while maintaining a level of systematic rigor, making them effective for communication and collaboration among design teams. \cite{Pahl07i}

\subsection{Searching for Working Principles}

