\chapter{Literature Review}
\label{ch:literaturereview}

\section{Methodic Product Development}
\label{sec:methodicproductdevelopment}
Methodic Product Development, by Pahl and Beitz, stress the importance of systematic design in product development \cite[9]{Pahl2007}. They differentiate between design science and methodology, with the latter being a practical approach based on scientific analysis and practical experience.

Pahl and Beitz describe the product development process as a series of stages, each with defined objectives and activities \cite[128-133]{Pahl2007}. The four main stages are:

\textbf{Planning and Task Clarification:} The process starts with planning and defining tasks, often in collaboration with the marketing or dedicated planning team. It is essential to thoroughly understand the task, whether from a product proposal or a customer request. This step involves gaining detailed insights into requirements, constraints, and their importance, forming a solid foundation for the next steps.

\textbf{Conceptual Design:} This phase involves defining the necessary functions, establishing working principles, and integrating them into a working structure. Ultimately, this leads to creating a fundamental solution that embodies the core of the design vision.

\textbf{Embodiment Design:} Guided by technical and economic considerations, the physical structure is defined. Various initial layouts are assessed to identify design strengths and weaknesses, ultimately leading to selecting the most promising version.

\textbf{Detail Design:} In this phase, precise arrangements, dimensions, materials, and other aspects are defined. Product documentation is created, including drawings, parts lists, and assembly instructions.

Figure \ref{fig:pahlprocess} shows the stages involved in Pahl and Beitz's design process.

\begin{figure}[ht!]
  \centering
  \includegraphics[width=0.75\textwidth]{texs/Part1/chapter1/image/pahlprocess.png}
  \caption{Pahl and Beitz's Design Process \cite[130]{Pahl2007}}
  \label{fig:pahlprocess}
\end{figure}

\section{Fused Deposition Modeling}
\label{sec:fused_deposition_modeling}

Fused deposition modeling (FDM) is a widely used technique in additive manufacturing, particularly in 3D printing. It offers several advantages that contribute to its popularity in various industries. One of the main advantages of FDM is its ability to reproduce complex geometries with high precision and accuracy \cite{Gordeev18}.

This method makes it suitable for creating intricate and customized designs that may not be achievable with traditional manufacturing methods. Additionally, FDM is a cost-effective process as it reduces material waste by only depositing the necessary amount of material layer by layer \cite{Gordeev18}, which not only saves costs but also promotes sustainability in manufacturing.

Common plastics used in FDM include acrylonitrile butadiene styrene (ABS), polylactic acid (PLA), and polyethylene terephthalate (PET) \cite{Teamm17}. These materials have different mechanical properties, advantages, and disadvantages, making them suitable for different applications.

\subsection{Original Prusa i3 MK3S+}
\label{subsec:prusa_slicer_mk3s}
The Original Prusa i3 MK3S+ is an FDM 3D printer designed for desktop use, ideal for tasks such as rapid prototyping and small-scale production. With a build volume of 250 mm x 210 mm x 200 mm, it can achieve layer heights ranging from 0.05 mm to 0.35 mm \cite{Prusa}. This printer has a heated bed and is compatible with various materials such as PLA, ABS, PETG, and nylon \cite{Prusa}. The default nozzle size is 0.4 mm, although alternate sizes can be utilized based on specific printing needs.

This 3D printer is accessible to students and faculty at the University of Applied Sciences Brandenburg and will play a pivotal role in producing the prototype for this work. Figure \ref{fig:prusa_slicer_mk3} visually represents the Original Prusa i3 MK3S+, located within the University of Applied Sciences Brandenburg Workshop.

\begin{figure}
  \centering
  \includegraphics[height=8cm]{texs/Part1/chapter1/image/prusa.jpg}
  \caption{Original Prusa i3 MK3S+}
  \label{fig:prusa_slicer_mk3}
\end{figure}

\subsection{PrusaSlicer}
\label{subsec:prusa_slicer}

PrusaSlicer is a free and open-source slicing software that converts 3D models into G-code, a language instructing the 3D printer on printing the object. It is compatible with a wide range of 3D printers and supports a variety of filament materials. PrusaSlicer offers many features that allow users to customize the printing process to suit their needs.

PrusaSlicer's ability to adjust printing parameters is crucial. These parameters include layer height, infill density, and print speed. Layer height determines the thickness of each layer of the printed object. Infill density refers to the material used to fill the object's inside. Print speed is the rate at which the printer moves while printing the object. Adjusting these parameters can help achieve the desired quality of the final product.

PrusaSlicer offers an essential function of adding support to the 3D prints. Supports are structures that print alongside the object to provide extra stability during printing. They prevent any potential collapse or deformation of the object while printing. Supports can be added manually or automatically depending on the complexity of the printed object.

This software can also generate a preview of the printed object, which lets users visualize the result. Additionally, PrusaSlicer provides an estimate of the amount of filament required for the printing process and the duration of the printing process. Figure \ref{fig:prusa_slicer} shows a screenshot of PrusaSlicer. The left side of the window shows the 3D model of the object, while the right side shows the various parameters that can be adjusted.

\begin{figure}
  \centering
  \includegraphics[width=0.8\linewidth]{texs/Part1/chapter1/image/prusaslicer.png}
  \caption{Example View of PrusaSlicer}
  \label{fig:prusa_slicer}
\end{figure}


\subsection{Printing Cost}
\label{subsec:printing_cost}

To perform a cost analysis of the 3D printing process, we will consider the following parameters:

\begin{itemize}
  \item Material Cost ($C_m$)
  \item Energy Cost ($C_e$)
\end{itemize}

Equation \ref{eq:material_cost} shows the formula used to calculate the material cost. This formula involves multiplying the mass of filament used ($m_{fil}$) by the cost of filament per kilogram($C_{fil}$). The cost of the filament is dependent on the type of material used. We will use PLA for this project, which costs 29.99 €/kg \cite{PrusaCost}.

Energy cost refers to the electricity cost of the printing process and is calculated using Equation \ref{eq:energy_cost}. The printing duration ($t_p$) is estimated directly from the PrusaSlicer software, while the power consumption ($P$) of the printer is estimated to be about 0.08 kW \cite{Prusa}. By observing the average price of electricity in Germany for the year 2022 \cite{NordPool}, the electricity price ($C_{el}$) is estimated at 0.235 €/kWh.

Equation \ref{eq:printing_cost} shows the formula for calculating the cost of 3D printing.

\begin{equation}
  \label{eq:material_cost}
  C_{m} = m_{fil}\cdot C_{fil}
\end{equation}

\begin{equation}
  \label{eq:energy_cost}
  C_{e} = t_{p}\cdot C_{el}\cdot P
\end{equation}

\begin{equation}
  \label{eq:printing_cost}
  C_{print} = C_{m} + C_{e}
\end{equation}



