\addto\captionsenglish{\renewcommand{\abstractname}{Abstract}}
\addto\captionsgerman{\renewcommand{\abstractname}{Kurzfassung}}

\selectlanguage{english}
\begin{abstract}
    This research presents the development of a handheld speed measurement device offering a cost-effective alternative to conventional speed pistols. A prototype has been engineered by integrating affordable computational components and high-quality cameras and employing 3D printing technology. The design process was guided by VDI Guideline 2221. Multiple variants were generated and evaluated against their technical and economic rating to find the optimal design. The selected variant was then produced by additive manufacturing and tested for functionality.

    Keywords: Speed Mesurement, 3D Printing, VDI Guideline 2221
\end{abstract}

\selectlanguage{german}
\begin{abstract}
    Die vorliegende Arbeit befasst sich mit der Entwicklung eines mobilen Geschwindigkeitsmessgeräts, das eine kostengünstige Alternative zu herkömmlichen Geschwindigkeitspistolen darstellt. Ein Prototyp wurde durch die Integration von kostengünstigen Rechenkomponenten und hochwertigen Kameras sowie durch den Einsatz von 3D-Drucktechnologie entwickelt. Der Entwurfsprozess orientierte sich an der VDI-Richtlinie 2221. Es wurden mehrere Varianten erstellt und anhand ihrer technischen und wirtschaftlichen Bewertung bewertet, um das optimale Design zu finden. Die ausgewählte Variante wurde dann mittels additiver Fertigung hergestellt und auf ihre Funktionalität getestet.

    Schlüsselwörter: Geschwindigkeitsmessung, 3D-Druck, VDI-Richtlinie 2221
\end{abstract}

\selectlanguage{english}