%%%%%%%%%%%%%%%%%%%%%%%%%%%%%%%%%%%%%%%%%%%%%%%%%%%%%%%%%%%%%%%%%%%%%%%%%%%%%%%%
% REQUIRED PACKAGES
%%%%%%%%%%%%%%%%%%%%%%%%%%%%%%%%%%%%%%%%%%%%%%%%%%%%%%%%%%%%%%%%%%%%%%%%%%%%%%%%
\usepackage{a4wide}
\usepackage[T1]{fontenc}
\usepackage[utf8]{inputenc}
% \usepackage[ngerman]{babel}
\usepackage[english, german]{babel}
\usepackage{ae,aecompl}
\usepackage{ae}
\usepackage{graphicx}
% \usepackage[
% 	backend=biber,
% 	style=alphabetic,
% ]{biblatex}
\usepackage{bibgerm}
% \usepackage{mathptmx} %Schriftart Times New Roman

\usepackage{dsfont}

\usepackage{setspace}
\onehalfspace


\usepackage{caption}
\usepackage{subcaption}
\usepackage{fancyhdr}

\usepackage{amssymb}
\usepackage{amsmath}
\usepackage{amsfonts}

\usepackage{makeidx}
\usepackage{nomencl}
\usepackage{hyperref}
\usepackage{float}
\usepackage{color}
\usepackage[dvipsnames]{xcolor}
\usepackage{listings} %zur Einbindung von anderen Codes

\usepackage{braket}
\usepackage{url}

\usepackage{mathpazo}
\usepackage[scaled=.95]{helvet}
\usepackage{courier}
% \renewcommand\familydefault{\sfdefault}

% use paragraph spacing
\usepackage{parskip}
\usepackage[headheight=15mm]{geometry}

\usepackage{titlesec}
\titleformat{\chapter}[block]{\normalfont\sffamily\huge\bfseries}{\thechapter}{0.5 em}{}
\titleformat{\section}[block]{\normalfont\sffamily\Large\bfseries}{\thesection}{0.5 em}{}
\titleformat{\subsection}[block]{\normalfont\sffamily\large\bfseries}{\thesubsection}{0.5 em}{}
\titlespacing{\chapter}{0pt}{3.0em}{2.0em}
\titlespacing{\section}{0pt}{2.0em}{2.0em}
\titlespacing{\subsection}{0pt}{2.0em}{1.5em}


% \usepackage[style=chicago-authordate,natbib=true,backend=biber]{biblatex}



\hypersetup{
    % bookmarks=true,								% Lesezeichen erzeugen
    bookmarksopen=false,					% Lesezeichen ausgeklappt
    bookmarksnumbered=true,				% Anzeige der Kapitelzahlen am Anfang der Namen der Lesezeichen
    pdfstartpage=1,							% Seite, welche automatisch ge�ffnet werden soll
    pdftitle={Titel},
    % Titel des PDF-Dokuments
    pdfauthor={Autor},	% Autor(Innen) des PDF-Dokuments
    pdfsubject={Betreff},	% Inhaltsbeschreibung des
    pdfkeywords={Schl�sselw�rter},
    % Stichwortangabe zum PDF-Dokument
    breaklinks=true,							% erm�glicht einen Umbruch von URLs
    colorlinks=true,							% Einf�rbung von Links
    linkcolor=myBlue,							% Linkfarbe: blau
    anchorcolor=blue,						% Ankerfarbe: schwarz
    citecolor=myRed, 							% Literaturlinks: schwarz
    filecolor=black,							% Links zu lokalen Dateien: schwarz
    menucolor=black, 							% Acrobat Men� Eintr�ge: schwarz
    % pagecolor=myBlue, 							% Links zu anderen Seiten im Text: schwarz
    urlcolor=myBlue,							% URL-Farbe: blau
    %backref=true,
    % pagebackref=false,
    pdfcenterwindow=true,
    pdfnewwindow=true,
    pdffitwindow=true,
    pdfstartview=FitH,
    pdfpagemode=UseOutlines
}

\definecolor{myBlue} {rgb}{0.423529412,0.388235294,0.725490196}
\definecolor{myRed} {cmyk}{0.02,0.88,0.99,0.0}
\definecolor{myBlue2} {cmyk}{0.681,0.511,0,0.471}
\definecolor{myTablehead} {cmyk}{0.57,0.19,0,0}

\let\abbrev\nomenclature
\renewcommand{\nomname}{List of Abbreviations}
\setlength{\nomlabelwidth}{.3\hsize}
\renewcommand{\nomlabel}[1]{#1 \dotfill}
\setlength{\nomitemsep}{-\parsep}
\makenomenclature


% box frame size
\fboxsep=1pt
\fboxrule=1pt

\usepackage{cleveref}
\usepackage{pdflscape}
\usepackage{pdfpages}

\def\UrlBreaks{\do\/\do-}
